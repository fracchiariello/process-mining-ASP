%!TEX root = main.tex

\section{ASP encodings}
We report three examples of Clingo ASP programs for Log Generation,
Query Checking and Conformance Checking.

\subsection{Event Log Generation}
Consider the instance of 
Log Generation consisting of the set of 
constraints $\Phi=\{\phi_1,\phi_2\}$, with:
\begin{enumerate}
 \item $\phi_1=\always((a1 \land number<5)\limp F a2)$;
 \item $\phi_2=F (a1 \land number < 3)$.
\end{enumerate}
%%
The automaton for $\phi_1$ is depicted in Fig.~\ref{fig:response},
where:
\begin{itemize}
	\item $\varphi_1=a1 \land number<5$;
	\item $\varphi_2=a2$;
	\item $\varphi_3=\lnot \varphi_1$;
	\item $\varphi_4=\lnot\varphi_2$.
\end{itemize}
The automaton for $\phi_2$ is depicted in Fig.~\ref{fig:presence},
where:
\begin{itemize}
	\item $\varphi_1=(a1 \land number < 3)$;
	\item $\varphi_2=\lnot\varphi_1$;
	\item $\varphi_3=\true$;
\end{itemize}
%%
\begin{figure}
	\begin{center}
		\begin{tikzpicture}
		    \node[state, initial, accepting] (s0) {$s_0$};
		    \node[state, right of=s0, xshift=.5cm] (s1) {$s_1$};
		    \draw
			(s0) edge[loop above] node{$\varphi_3$} (s0)
	            	(s0) edge[bend left, above] node{$\varphi_1$} (s1)
	            	(s1) edge[loop above] node{$\varphi_4$} (s1)
	            	(s1) edge[bend left, above] node{$\varphi_2$} (s0);
		\end{tikzpicture}
	\end{center}
\caption{Automaton for constraint $\phi_1$.\label{fig:response}}
\end{figure}
%%
\begin{figure}
	\begin{center}
		\begin{tikzpicture}
		    \node[state, initial] (s0) {$s_0$};
		    \node[state, right of=s0, xshift=.5cm, accepting] (s1) {$s_1$};
		    \draw
			(s0) edge[loop above] node{$\varphi_2$} (s0)
	            	(s0) edge[above] node{$\varphi_1$} (s1)
	            	(s1) edge[loop above] node{$\varphi_3$} (s1);
		\end{tikzpicture}
	\end{center}
\caption{Automaton for constraint $\phi_2$.\label{fig:presence}}
\end{figure}
%%
For the length of the trace(s) to return, we use the parameter 
$t$, which can be input provided when Clingo is launched.
We report the actual code for Clingo below.

\begin{small}
\begin{aspcode}
\noindent
\% PROBLEM INSTANCE\\

\noindent
\% Command~line-parameters:\\
\% t:~trace~length\\ 
(use option -c t=<value> to assign)\\

\noindent
\% Activities:\\
activity(a1).
activity(a2).
activity(a3).
activity(a4).
activity(a5).
activity(a6).
activity(a7).
activity(a8).
activity(a9).
activity(10).\\

\noindent
\% Activity attributes:\\ 
\% has\_attribute(activity, attribute name)\\
has\_attribute(a1,categorical).
has\_attribute(a1,number).\\

\noindent 
\% Attribute types:\\ 
\% value(attribute name, allowed value):\\
value(number,0..100).\\
value(categorical,c1). 
value(categorical,c2). 
value(categorical,c3).\\

\noindent
\% Automata:\\
\% Every automaton has associated a unique index\\
\% initial(automaton index, initial state)\\
\% accepting(automaton index, accepting state)\\
\% transition(automaton index, src state, 
\%~~~~~~~~~~~~~~~~~~~~formula index, dest state).\\
\% every formula has associated an index and\\
\% is associated to an automaton\\
\% (through automaton index)\\
\% holds(automaton index, formula index,\\
\%~~~~~~~~~~~~~~~~~~~~~~~~~~~~~trace index)\\

\noindent
\% phi1=G((a1 and number<5)-> F a2)\\
initial(1,s0).\\
accepting(1,s0).\\
transition(1,s0,1,s1).\\
holds(1,1,T) :- trace(a1,T),\\ 
\phantom{holds(1,1,T) :- tra}has\_value(number,V,T), V<5.
transition(1,s1,2,s0).\\
holds(1,2,T) :- trace(a2,T).\\
transition(1,s0,3,s0).\\
holds(1,3,T) :- not holds(1,1,T), time(T).\\
transition(1,s1,4,s1).\\
holds(1,4,T) :- trace(A,T), activity(A), A != a2.\\

\noindent
\% F a1 and number < 3\\
initial(2,s0).\\
accepting(2,s1).\\
transition(2,s0,1,s0).\\
holds(2,1,T) :- not holds(2,2,T), time(T).\\
transition(2,s0,2,s1).\\
holds(2,2,T) :- trace(a1,T), has\_value(number,V,T), V<3.\\
transition(2,s1,3,s1).\\
holds(2,3,T) :- time(T).\\

\noindent
\% Time points\\
time(0..t). \\

\noindent
\% PROBLEM ENCODING\\

\noindent
\% Exactly one activity per trace position\\ 
\% (0,...,t-1)\\
\{trace(A,T) : activity(A)\} = 1 :- time(T), T<t.\\

\noindent
\% Exactly one assigned value per attribute\\ 
\% associated with each activity\\
\{has\_value(K,V,T) : value(K,V)\} = 1 :- trace(A,T), has\_attribute(A,K).\\

\noindent
\% Initial state\\
cur\_state(I,S,0) :- initial(I,S).\\

\noindent
\% Transitions\\
cur\_state(I,S',T) :- cur\_state(I,S,T-1), transition(I,S,F,S'), holds(I,F,T-1).\\

\noindent
\% Trace must be accepted by all automata\\
\% Every automaton must be in at least one\\ 
\% accepting state at end of run\\
:- {cur\_state(I,S,T): not accepting(I,S), trace\_length(T)} = 0.\\

\noindent
\% Show solution\\
\#show trace/2.\\
\#show has\_value/3.\\
\end{aspcode}
\end{small}

\subsection{Query Checking}
As explained in the paper, the input of Query Checking 
includes many traces, each of which is uniquely identified by an index.
We show the input for the following two traces:
\begin{itemize}
	\item $\tau_1=a2()~a1(2,c3)~a2()$;
	\item $\tau_2=a1(2,c3)~a1(1,c2)~a1(2,c2)~a2()$,
\end{itemize}
%%
where the first attribute of activity $a1$ is named $number$ and the 
second is named $categorical$. Activity $a2$ does not contain any attribute.
We remind that since many traces are present, a further index must be added,
wrt the encoding for Log Generation, to various predicates, in order 
to indicate the trace that the predicate refers to. We denote such a predicate 
with the symbol {\asp J}.

As input constraints, we use the following event formulas with 
activity variables:
\begin{itemize}
	\item $\phi_1=\always((?A1 \land number<5) \limp F ?A2)$;
	\item $\phi_2=F (?A1 \land number < 3)$.
\end{itemize}
%%
The automaton corresponding to $\phi_1$ is depicted in Fig.~\ref{fig:response},
with: 
\begin{itemize}
	\item $\varphi_1=?A1 \land number<5$;
	\item $\varphi_2=?A2$;
	\item $\varphi_3=\lnot \varphi_1$;
	\item $\varphi_4=\lnot\varphi_2$.
\end{itemize}
%%
The automaton corresponding to $\phi_2$ is depicted in Fig.~\ref{fig:presence},
where:
\begin{itemize}
	\item $\varphi_1=(?A1 \land number < 3)$;
	\item $\varphi_2=\lnot\varphi_1$;
	\item $\varphi_3=\true$;
\end{itemize}
To account for the presence of many traces, predicate 
{\asp holds}, which models satisfaction of event formulas at
a given time point, contains an additional parameter representing 
the trace identifier. This allows to parameterize satisfaction of a 
formula wrt to the position of the trace.

Finally, as discussed in the paper, each variable $?A$ is 
modeled in the ASP program with object {\asp varA} in predicate
{\asp variable}, while the assignment to a variable {\asp varA} is 
modeled with the binary predicate {\asp assignment(varA,val)}, where 
{\asp val} is the value assigned to {\asp varA}.

The resulting encoding is shown below.\\

\begin{small}
\begin{aspcode}
\noindent
\% PROBLEM INSTANCE\\

\noindent
\% traces\\
\% trace(id,activity,position)\\

\noindent
\% tau1=a2() a1(2,c3) a2()\\
trace(1,a2,0).\\
trace(1,a1,1).\\
has\_value(1,integer,2,1).\\
has\_value(1,categorical,c3,1).\\
trace(1,a2,2).\\
trace\_length(1,3).\\

\noindent
\% tau2=a1(2,c3) a1(1,c2) a1(2,c2) a2()\\
trace(2,a1,0).\\
has\_value(2,integer,2,0).\\
has\_value(2,categorical,c3,0).\\
trace(2,a1,1).\\
has\_value(2,integer,1,1).\\
has\_value(2,categorical,c2,1).\\
trace(2,a1,2).\\
has\_value(2,integer,1,2).\\
has\_value(2,categorical,c2,2).\\
trace(2,a2,3).\\
trace\_length(2,4).\\

\noindent\% Activities\\
\% Derived from trace\\
activity(A) :- trace(\_,A,\_).\\

\noindent
\% Activity attributes:\\
\% has\_attribute(activity, attribute name)\\
has\_attribute(a1,categorical).\\
has\_attribute(a1,number).\\

\noindent
\% Attribute types:\\
\% value(attribute name, allowed value):\\
value(number,0..100).\\
value(categorical,c1).\\
value(categorical,c2).\\
value(categorical,c3).\\

\noindent
\% Automata encoding:\\
\% Almost same as Log Generation\\
\% differences: holds include trace id)\\
\% holds(automaton index, formula index,\\ 
\% \phantom{holds(au} trace id, time point)\\

\noindent
\% G((?A1 and number<5) -> F ?A2)\\
variable(varA1).\\
variable(varA2).\\
initial(1,s0).\\
accepting(1,s0).\\
transition(1,s0,1,s1).\\
holds(1,1,J,T) :- trace(J,A,T), V<5,\\ 
\phantom{ho}assignment(varA1,A), has\_value(J,integer,V,T).\\ 
transition(1,s1,2,s0).\\
holds(1,2,J,T) :- trace(J,A,T),\\ 
\phantom{holds(1,2,J,T) :- }assignment(varA2,A).\\
transition(1,s0,3,s0).\\
holds(1,3,J,T) :- not holds(1,1,J,T), time(J,T).\\
transition(1,s1,4,s1).\\
holds(1,4,J,T) :- not holds(1,2,J,T), time(J,T).\\

\noindent 
\% F ?A1 and number < 3\\
initial(2,s0).\\
accepting(2,s1).\\
transition(2,s0,1,s0).\\
holds(2,1,J,T) :- not holds(2,2,J,T), time(J,T).\\ 
transition(2,s0,2,s1).\\
holds(2,2,J,T) :- trace(J,A,T), V<3,\\ 
\phantom{ho}assignment(varA1,A), has\_value(J,integer,V,T).\\
transition(2,s1,3,s1).\\
holds(2,3,J,T) :- time(J,T).\\

\noindent
\%time points\\
time(J,0..T) :- trace\_length(J,T).\\

\noindent
\% PROBLEM ENCODING\\

\noindent
\% exactly one activity per trace point\\
\{trace(I,A,T) : activity(A)\} = 1 :- time(I,T),\\
\phantom{\{trace(I,A,T) : }T < L, trace\_length(I,L).\\

\noindent
\%initial state (J is trace identifier):\\
cur\_state(I,J,S,0) :- initial(I,S), trace(J,\_,\_).\\

\noindent
\%transitions (J is trace identifier):
cur\_state(I,J,S',T) :- cur\_state(I,J,S,T-1),\\
\phantom{cur\_state}transition(I,S,F,S'), holds(I,F,J,T-1),\\
\phantom{cur\_state}trace(J,\_,\_).\\

\noindent
\%exactly one assigned value per attribute\\
{has\_value(J,K,V,T) : value(K,V)} = 1 :- \\
\phantom{has\_value}trace(J,A,T), has\_attribute(A,K).\\

\noindent
\%exactly one assigned value per variable\\
\{assignment(Var,Val) : activity(Val)\} = 1 :- \\
\phantom{\{assignment(Var,Val)}variable(Var).\\

\noindent
:- cur\_state(I,J,S,L), trace\_length(J,L),\\
\phantom{:- }not accepting(I,S).\\

\noindent
\% solution\\
\#show trace/3.\\
\#show has\_value/4.\\
\#show assignment/2.
\end{aspcode}
\end{small}

\subsection{Conformance Checking}
As explained in the paper, this is a special case of query checking where 
the constrains are variable free, i.e., they are fully specified. 
To obtain the encoding, it is enough to replace the encoding of the automata 
with one that is fully specified, and remove the predicates related to variables, 
i.e., {\asp variable} and {\asp assignment}.
No predicates are returned as solution, as the instance is solved if the 
specification is satisfiable.

For completeness, we report the full 
ASP encoding for the set of traces used in the example of Conformance Checking 
and the constraints (automata) used for Log Generation.\\

\begin{small}
\begin{aspcode}
\noindent
\% PROBLEM INSTANCE\\

\noindent
\% traces\\
\% trace(id,activity,position)\\

\noindent
\% tau1=a2() a1(2,c3) a2()\\
trace(1,a2,0).\\
trace(1,a1,1).\\
has\_value(1,integer,2,1).\\
has\_value(1,categorical,c3,1).\\
trace(1,a2,2).\\
trace\_length(1,3).\\

\noindent
\% tau2=a1(2,c3) a1(1,c2) a1(2,c2) a2()\\
trace(2,a1,0).\\
has\_value(2,integer,2,0).\\
has\_value(2,categorical,c3,0).\\
trace(2,a1,1).\\
has\_value(2,integer,1,1).\\
has\_value(2,categorical,c2,1).\\
trace(2,a1,2).\\
has\_value(2,integer,1,2).\\
has\_value(2,categorical,c2,2).\\
trace(2,a2,3).\\
trace\_length(2,4).\\

\noindent\% Activities\\
\% Derived from trace\\
activity(A) :- trace(\_,A,\_).\\

\noindent
\% Activity attributes:\\
\% has\_attribute(activity, attribute name)\\
has\_attribute(a1,categorical).\\
has\_attribute(a1,number).\\

\noindent
\% Attribute types:\\
\% value(attribute name, allowed value):\\
value(number,0..100).\\
value(categorical,c1).\\
value(categorical,c2).\\
value(categorical,c3).\\

\noindent
\% Automata encoding:\\
\% Almost same as Log Generation\\
\% differences: holds include trace id)\\
\% holds(automaton index, formula index,\\ 
\% \phantom{holds(au} trace id, time point)\\

\noindent
\% G((a1 and number<5) -> F a2)\\
initial(1,s0).\\
accepting(1,s0).\\
transition(1,s0,1,s1).\\
holds(1,1,J,T) :- trace(J,a1,T), V<5,\\ 
\phantom{ho}, has\_value(J,integer,V,T).\\ 
transition(1,s1,2,s0).\\
holds(1,2,J,T) :- trace(J,a2,T).\\
transition(1,s0,3,s0).\\
holds(1,3,J,T) :- not holds(1,1,J,T), time(J,T).\\
transition(1,s1,4,s1).\\
holds(1,4,J,T) :- not holds(1,2,J,T), time(J,T).\\

\noindent 
\% F a1 and number < 3\\
initial(2,s0).\\
accepting(2,s1).\\
transition(2,s0,1,s0).\\
holds(2,1,J,T) :- not holds(2,2,J,T), time(J,T).\\ 
transition(2,s0,2,s1).\\
holds(2,2,J,T) :- trace(J,a1,T), V<3,\\ 
\phantom{ho}has\_value(J,integer,V,T).\\
transition(2,s1,3,s1).\\
holds(2,3,J,T) :- time(J,T).\\

\noindent
\%time points\\
time(J,0..T) :- trace\_length(J,T).\\

\noindent
\% PROBLEM ENCODING\\

\noindent
\% exactly one activity per trace point\\
\{trace(I,A,T) : activity(A)\} = 1 :- time(I,T),\\
\phantom{\{trace(I,A,T) : }T < L, trace\_length(I,L).\\

\noindent
\%initial state (J is trace identifier):\\
cur\_state(I,J,S,0) :- initial(I,S), trace(J,\_,\_).\\

\noindent
\%transitions (J is trace identifier):
cur\_state(I,J,S',T) :- cur\_state(I,J,S,T-1),\\
\phantom{cur\_state}transition(I,S,F,S'), holds(I,F,J,T-1),\\
\phantom{cur\_state}trace(J,\_,\_).\\

\noindent
\%exactly one assigned value per attribute\\
{has\_value(J,K,V,T) : value(K,V)} = 1 :- \\
\phantom{has\_value}trace(J,A,T), has\_attribute(A,K).\\

\noindent
\%exactly one assigned value per variable\\
\{assignment(Var,Val) : activity(Val)\} = 1 :- \\
\phantom{\{assignment(Var,Val)}variable(Var).\\

\noindent
:- cur\_state(I,J,S,L), trace\_length(J,L),\\
\phantom{:- }not accepting(I,S).\\

\end{aspcode}
\end{small}
